\begin{yl}{5}{Kingituste andmebaas}{baas}{5 sek / 10 sek}{60 punkti}
  \emph{Idee ja teostus: Heno Ivanov, lahenduse selgitus: Tähvend Uustalu}

  Olgu antud mingi nimekiri sõnedest.
  \emph{Trie'ks} nimetatakse andmestruktuuri, milles on tipp iga sõne kohta, mis
  on mõne nimekirjas oleva sõne prefiks. Sõnele $s$ vastavast tipust on tähega $c$ märgistatud serv sõnele $t$
  vastavasse tippu, kui sõnele $s$ tähe $c$ lõppu lisamise annab sõne $t$.

  Näiteks vastab nimekirjale ``piim'', ``pidu'', ``asi'', ``appi'' alltoodud trie.

  \begin{center}
    \begin{tikzpicture}
      \begin{scope}[every node/.style={circle, draw, minimum size=10pt}]
        \node (root) at (0, 0) {};
        \node (p) at (1, -1) {};
        \node (pi) at (1, -2) {};
        \node (pid) at (0, -3) {};
        \node (pidu) at (0, -4) {};
        \node (pii) at (2, -3) {};
        \node (piim) at (2, -4) {};
        \node (a) at (-1, -1) {};
        \node (ap) at (-2, -2) {};
        \node (app) at (-2, -3) {};
        \node (appi) at (-2, -4) {};
        \node (as) at (-1, -2) {};
        \node (asi) at (-1, -3) {};
      \end{scope}
      \begin{scope}[>={Stealth[black]},
          every node/.style={fill=white,circle,scale=0.5}]
        \draw[->] (root) -- (p) node[midway] {p};
        \draw[->] (p) -- (pi) node[midway] {i};
        \draw[->] (pi) -- (pid) node[midway] {d};
        \draw[->] (pid) -- (pidu) node[midway] {u};
        \draw[->] (pi) -- (pii) node[midway] {i};
        \draw[->] (pii) -- (piim) node[midway] {m};
        \draw[->] (root) -- (a) node[midway] {a};
        \draw[->] (a) -- (ap) node[midway] {p};
        \draw[->] (ap) -- (app) node[midway] {p};
        \draw[->] (app) -- (appi) node[midway] {i};
        \draw[->] (a) -- (as) node[midway] {s};
        \draw[->] (as) -- (asi) node[midway] {i};
      \end{scope}
    \end{tikzpicture}
  \end{center}

  Üks viis trie ja sellesse sõne lisamise realiseerimiseks on järgmine:

  \begin{lstlisting}
struct Node { // tipp
  map<char, Node> next;
};

void add_word(Node & root, const string & word) {
  Node *curr = &root;
  for (auto c : word) {
    // next[c] poole pöördumine lisab uue elemendi,
    // kui seda veel olemas ei ole
    curr = &curr->next[c];
  }
}
\end{lstlisting}

  Mõtleme ülesandest nii: tries on sõnele vastav tipp parajasti siis, kui ta on prefiks vähemalt ühele
  sõnale, millest trie moodustatud on. Kui me ehitame iga lapse kingisoovidest trie, siis
  otsime pikimat sõnet, millele vastav tipp leidub igas tries. Teisisõnu otsime sügavaimat tippu
  kõikide triede ühisosas.

  Kahe trie ühisosa saab leida näiteks nii:
  \begin{lstlisting}
// märk '&' tüübi järel tähendab, et funktsioon saab argumendiks antud
// struktuuri aadressi ja andmetest koopiat ei tehta
Node intersect(Node & a, Node & b) {
  Node res;
  for (auto & [c, _] : a.next) {
    // käime läbi kõik tipu a alluvad; kui tipul b on ka sama tähega alluv,
    // siis arvutame rekursiivselt nende alampuude ühisosa ja lisame selle
    // ühisosa-triesse
    if (b.next.contains(c)) {
      res.next[c] = intersect(a.next[c], b.next[c]);
    }
  }
  return res;
}
  \end{lstlisting}

  Tähelepanu tuleb pöörata ka mälulimiidile. Kuigi see on antud ülesandes 512~MB --- suurem kui harilik
  256~MB --- võib hooletu lahendus ka seda ületada.
  Sisendi maksimaalne maht on 64~MB ehk kingisoovide pikkuste summa võib olla kuni $64 \cdot 2^{20}$.
  Seega, kui iga lapse kingisoovidest ehitada trie ja kõik need tried samaaegselt mälus on,
  võib nendel triedel ka kokku olla $64 \cdot 2^{20}$ tippu. Kui iga tipp võtab näiteks 10 baiti,
  siis ongi juba 640~MB kulutatud (ja 512~MB piir ammu ületatud).
  Üks lahendus on alati hoida meeles globaalset ühisosa-triet, aga iga lapse triet vaid ühes iteratsioonis.
  \begin{lstlisting}
Node file_trie;
for (int i = 0; i < n; ++i) {
  // line_trie on deklareeritud ploki sees - see tähendab, et iga line_trie
  // kustutatakse kohe, kui iteratsioon on läbi, ja järgmiseks iteratsiooniks
  // tehakse uus tühi trie
  Node line_trie;
  for (int j = 0; j < m; ++j) {
    string word;
    cin >> word;
    add_word(line_trie, word);
  }
  if (i == 0) {
    file_trie = line_trie;
  } else {
    file_trie = intersect(file_trie, line_trie);
  }
}
\end{lstlisting}

\textit{Alternatiivne lahendus: Marko Tsengov}

Oletame, et me teame päringuks olnud sõne pikkust $\ell$.
Sellisel juhul saame tulemused rida-rea haaval läbi käia ning hoida meeles, millised võimalikud päringud eelnevate ridade põhjal olla võisid.
Iga uue rea puhul saame leida päringud, mis just sellele reale oleks sobinud, ning leida kõigi seni sobinud (iga sõne vastava pikkusega prefiks) ja sellele reale sobinud päringute ühisosa.
Sellise jooksva ühisosa meeles hoidmiseks sobib hästi hulga (\textit{set}) andmestruktuur, kust saab kiiresti (keerukusega $O(1)$ või $O(\log N)$) elemente pärida ning kustudada.

Et esimesel real on $M$ sõnet, peame korraga meeles hoidma kuni $M$ erinevat päringut, kuna ridade kaupa edasi minnes saab sobivaid päringuid ainult vähemaks jääda.
Ühisosa uuendamiseks võime kõik eelnevate ridade jaoks sobinud päringud läbi käia ning kontrollida, kas need sobivad ka uuele reale.
Et sel real oli niikuinii juba $M$ sõnet, mille sisselugemiseks kulus $M$ operatsiooni, pole nende kõigi kohta hulka kuuluvuse kontrollimine sellega võrreldes kuigi aeganõudev.
Seega saame sobivad päringud leida kogu andmestikku vaid ühe korra läbides.

Siiski peame veel parandama asjaolu, et me ei tea $\ell$ väärtust.
Märkame, et kui pakume reaalsest väärtusest suurema pikkuse, siis eelolev algoritm töötab, kuid lõpuks ei leia ühtki sobivat päringut.
Samuti märkame, et kui pakume reaalsest väärtusest väiksema pikkuse, siis sama algoritm endiselt töötab, tagastades õige päringu vastava pikkusega prefiksi.

Eeltoodud omaduste põhjal saame $\ell$ väärtuse leida kahendotsingu abil või --- kuna lubatud sõnede pikkus on võrdlemisi väike --- ka iteratiivselt, proovides järjest suuremaid väärtusi, kuni sobivat päringut enam ei leidu.

\end{yl}
