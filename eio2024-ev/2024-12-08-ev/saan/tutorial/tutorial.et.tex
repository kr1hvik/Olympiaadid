\begin{yl}{6}{Tasakaalus jõulusaan}{saan}{2 sek / 10 sek}{100 punkti}
  \emph{Idee: Heno Ivanov, teostus ja lahenduse selgitus: Tähvend Uustalu}

  Alustel $A_1, A_2, \ldots, A_K$ olevate pakkide massikese on $(A_1, A_2, \ldots, A_K) / K$;
  saani keskpunkt on $(N + 1) / 2$. Seega on saan tasakaalus, kui
  $$ \frac{A_1 + A_2 + \cdots + A_K}{K} = \frac{N + 1}{2} $$
  ehk
  $$ A_1 + A_2 + \cdots + A_K = \frac{K(N + 1)}{2}. $$
  Näeme, et kui $N$ on paaris ja $K$ on paaritu, siis lahendit olla ei saa:
  vasak pool on täisarv, parem mitte. Muudel juhtudel leidub vähemalt mingigi
  tasakaalus konfiguratsioon: pooled kastid vasakusse serva, pooled paremale, kasutades vajaduse
  korral ka keskmist alust.

  Lahendame ülesande dünaamilise plaanimise abil. Selleks on palju variante, esitame siinkohal
  ühe. Keskendume minimaalse käikude arvu leidmisele --- võimaluste arvu loendamine tuleb sellest
  lahendusest loomulikul teel välja. Pakkide ümbertõstmise asemel mõtleme pakkide lisamisest ja
  eemaldamisest: käime alused vasakult paremale läbi; kaalume igalt aluselt paki eemaldamist
  (kui sellel hetkel on pakk) ja paki lisamist (kui sellel hetkel pakki ei ole).
  Lõpplahendis peavad lisatud pakkide ja eemaldatud pakkide arvud muidugi võrdsed olema.

  Hetke olekut peegeldab kolmik $(i, j, S)$, kus:
  \begin{xitem}
  \item $i$ on läbi vaadatud aluste arv;
  \item $j$ on eemaldatud pakkide arvu ja lisatud pakkide arvu vahe;
  \item $S$ on alustel $1 \ldots i$ olevate pakkide asukohtade summa.
  \end{xitem}

  Tähistagu $\Dp[i][j][S]$ minimaalset eemaldatud pakkide arvu (või samaväärselt,
  lisatud pakkide arvu). Alguses seame $\Dp[0][0][0] = 0$ ja iga muu kolmiku korral
  $\Dp[i][j][S] = \infty$. Nüüd arvutame tabeli varasemate väärtuste põhjal välja
  hilisemad väärtused.
  \begin{xitem}
  \item Iga $i = 1, \ldots, n$ korral:
    \begin{xitem}
    \item Iga $j = -n, \ldots, n$ korral:
      \begin{xitem}
      \item Iga $S = 0, \ldots, \frac{n (n + 1)}{2}$ korral:
        \begin{xitem}
        \item Kui alusel $i$ on pakk:
          \begin{xitem}
          \item Seame $\Dp[i][j][S + i] \gets \min(\Dp[i][j][S + i], \Dp[i - 1][j][S])$.
            (Kaalume mitte millegi tegemist ehk paki alusele jätmist. See viiks meid
            olekust $(i - 1, j, S)$ olekusse $(i, j, S + i)$: ühtegi pakki ei eemaldatud ega
            lisatud, aga pakkide asukohtade summa, mis arvestab nüüd ka alusega $i$, kasvab $i$
            võrra. Miinimumi võtame sellepärast, et tabelis võib juba ees olla parem
            lahend sama oleku jaoks.)
          \item Seame $\Dp[i][j + 1][S] \gets \min(\Dp[i][j + 1][S], \Dp[i - 1][j][S])$.
            (Kaalume aluselt paki eemaldamist; see viiks meid olekust $(i - 1, j, S)$ olekusse
            $(i, j + 1, S)$ --- eemaldatud pakkide arv kasvab ühe võrra, asukohtade summa jääb
            samaks.)
          \end{xitem}
        \item Kui alusel $i$ pakki pole:
          \begin{xitem}
          \item Seame $\Dp[i][j][S] \gets \min(\Dp[i][j][S], \Dp[i - 1][j][S]$ (kaalume mitte
            millegi tegemist).
          \item Seame $\Dp[i][j - 1][S + i] \gets \min(\Dp[i][j - 1][S + i], \Dp[i - 1][j][S] + 1)$
            (kaalume alusele paki panemist).
          \end{xitem}
        \end{xitem}
      \end{xitem}
    \end{xitem}
  \end{xitem}

  Lõppvastus on tabeli lahtris $\Dp[N][0][K (N + 1) / 2]$. Selline lahendus muidu töötab,
  aga siin on mõned nüansid.

  Esiteks on $91 \times 181 \times 4096$ tabel päris suur. Mälulimiiti mahub see küll, aga
  nii suure \texttt{array} või \texttt{vector} loomine võib ebaõnnestuda. Selle probleemi
  lahendamiseks piisab tähele panna, et meil ei ole vaja tervet tabelit meeles hoida.
  Vaja on ainult praeguse $i$ ja $i - 1$ tabelit.

  Teiseks on vaja tabelis kasutada negatiivseid indekseid. Selleks on mitu moodust; žürii
  lahenduses on see lahendatud nii:
  \begin{lstlisting}
template<typename T>
class OffsetArray {
  int offset;
  vector<T> vec;

public:
  OffsetArray (int _size, int _offset, T init) :
    offset(_offset), vec(_size, init) { }

  T& operator[] (int index) {
    return vec[offset + index];
  }
};
  \end{lstlisting}
  Siis \texttt{OffsetArray<int>(n, m, 0)} loob massiivilaadse toote, mis on alguses täidetud
  nullidega ja milles on kasutatavad indeksid $-m \ldots n - m - 1$.

  Esimene pool ülesandest --- tõstmiste arvu minimeerimine --- on seega lahendatud keerukusega $O(N^4)$.
  Siia tuleb nüüd lisada võimaluste loendamine. Selleks peame iga oleku kohta meeles,
  mitu võimalust on minimaalse sammude arvuga selle oleku juurde jõuda. Teeme teise tabeli:
  $\mathrm{ways}[i][j][S]$ loendab, mitu võimalust $\Dp[i][j][S]$ käiguga olekusse $(i, j, S)$
  jõudmiseks on juba leitud.
  Seda tabelit uuendame koos tabeliga $\Dp$. Oletame näiteks, et alusel $i$ on pakk ja kaalume mitte
  millegi tegemist. Ülalolev algoritm teeb siis
  $\Dp[i][j][S + i] \gets \min(\Dp[i][j][S + i], \Dp[i - 1][j][S])$. Enne seda uuendame tabelit
  $\mathrm{ways}$:
  \begin{xitem}
  \item kui $\Dp[i - 1][j][S] < \Dp[i][j][S + i]$, siis \textbf{seame} $\mathrm{ways}[i][j][S + i]$
    väärtuseks $\mathrm{ways}[i - 1][j][S]$;
  \item kui $\Dp[i - 1][j][S] = \Dp[i][j][S + i]$, siis \textbf{liidame} $\mathrm{ways}[i][j][S + i]$
    väärtusele $\mathrm{ways}[i - 1][j][S]$;
  \item kui $\Dp[i - 1][j][S] > \Dp[i][j][S + i]$, siis ei tee me midagi.
  \end{xitem}
\end{yl}
