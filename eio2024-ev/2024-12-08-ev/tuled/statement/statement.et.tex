% Idee: Sandra Schumann
% Teostus: Sandra Schumann

\documentclass[a4paper,11pt]{article}
\usepackage[et]{../../eio}

\begin{document}
\begin{ol}{\eio}{\ev 8.12.2024}{\yle}{}
\begin{yl}{1}{Jõulutuled}{tuled}{1 sekund}{10 punkti}

Juta ehtis oma maja üleni jõulutuledega. Need on küll ilusad, aga kulutavad palju elektrit. Jutal on raha vähe, aga ta tahab tulesid töös hoida nii kaua kui vähegi võimalik.

Juta veidras elektripaketis maksab tulede tööshoidmine kella 6:00 ja 18:00 vahel $X$ senti tunnis ning kella 18:00 ja 6:00 vahel $Y$ senti tunnis. Juta ei taha tulesid tööle panna enne kella 14:00, sest siis on ta koolis, ja tahab need kustutada hiljemalt kell 23:00, sest siis läheb ta magama. Ühe päeva jooksul on Jutal võimalik tulede tööshoidmisele kulutada kõige rohkem $N$ eurot. Leia, mitu tundi (maksimaalselt) saab Juta tulesid ühe päeva jooksul töös hoida.

\sis
Sisendi esimesel real on täisarv $X$ ($1 \le X \le 1\,000$), mis näitab, mitu senti tunnis maksab tulede tööshoidmine ajavahemikus 6:00 kuni 18:00.

Teisel real on täisarv $Y$ ($1 \le Y \le 1\,000$), mis näitab, mitu senti tunnis maksab tulede tööshoidmine ajavahemikus 18:00 kuni 6:00.

Kolmandal real on täisarv $N$ ($1 \le N \le 100$), mis näitab, mitu eurot on Jutal võimalik maksimaalselt ühe päeva jooksul tulede tööshoidmisele kulutada.

\val
Väljastada üks arv, mis näitab, mitu tundi saab Juta tulesid ühe päeva jooksul töös hoida. Pane tähele, et see ei tarvitse olla täisarv. Kui õige vastus on murdarv (``komaga arv''), siis tuleks see nii ka väljastada (vaata ka näidet). Väljastatud vastus ei tohi õigest vastusest erineda rohkem kui $0,005$ võrra.

\nde[0]{3cm}{3cm}

Kui Juta paneb tuled põlema mõned minutid enne kella 16:00, nii, et tuled põlevad $2,05556$ tundi hinnaga $90$ senti tunnis ja seejärel $5$ tundi hinnaga $63$ senti tunnis, siis kulutab ta ära $2,05556 \cdot 90 + 5 \cdot 63 \approx 500$ senti ehk $5$ eurot, hoides tulesid põlemas kokku $7,05556$ tundi. Pikemalt ei ole Jutal võimalik tulesid põlemas hoida.

\nde[1]{3cm}{3cm}

Juta saab tulesid põlemas hoida $9$ tundi, koolist koju jõudmisest kuni magama minemiseni, sest see kulutab kokku vaid $68$ senti. Ülejääva raha võib Juta kulutada jõulukinkidele.

\hnd
Selles ülesandes antakse punkte iga testi eest eraldi. Testid on jagatud gruppidesse, milles kehtivad järgmised lisatingimused:
\begin{xenum}
	\item (0 punkti) Ülesande tekstis olevad näited. Nende lahendamise eest punkte ei saa, aga nende hindamise tulemustest on näha, kas programm töötab serveris testides õigesti.
	\item (5 punkti) Vastus on täisarv.
	\item (5 punkti) Lisapiirangud puuduvad.
\end{xenum}

\end{yl}
\end{ol}
\end{document}
