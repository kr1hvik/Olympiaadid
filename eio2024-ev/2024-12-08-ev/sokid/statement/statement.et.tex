% Idee: Tähvend Uustalu
% Teostus: Birgit Veldi

\documentclass[a4paper,11pt]{article}
\usepackage[et]{../../eio}

\begin{document}
\begin{ol}{\eio}{\ev 8.12.2024}{\yle}{}
\begin{yl}{2}{Sokid}{sokid}{1 sekund}{20 punkti}

Juku vanavanemad armastavad talle jõuludeks sokke kinkida. Sellepärast on Jukul kogunenud sahtlisse palju sokke. Mõned sokid on aja jooksul ka ära kadunud.

Praegu on Jukul sokisahtel sassis: kõik sokid on paari panemata. Juku ema saaks seda nähes väga pahaseks ning ütleks päkapikkudele, et Jukule ei tohiks kommi tuua.
Juku tahab aga olla hea laps ning ema kojutulekuks kõik sokid paari panna.
Kaks sokki saab paari panna siis, kui need on sama värvi. Selleks otsustab Juku osad sokid üle värvida.
Kuna aega on vähe, tahab ta üle värvida võimalikult vähe sokke.

Leia, mitu sokki tuleks Jukul üle värvida, et nad kõik paari panna.

\sis
Sisendi esimesel real on täisarv $N$ ($1 \le N \le 100\,000$): sokkide arv sahtlis.

Järgmisel $N$ real on igaühel kirjas ühe soki värv ($1$ kuni $20$ väikest ladina tähte).

\val
Väljastada üks täisarv: mitu sokki peab Juku minimaalselt ära värvima, et kõik sokid paaridesse jaotada. Kui sokkide paaridesse jaotamine pole võimalik, siis väljastada $-1$.

\nde[0]{3cm}{3cm}

Kui Juku värviks näiteks punase soki kollaseks, saaks ta kollaste ja siniste sokkide paarid.

\nde[1]{3cm}{3cm}

Kolme sokki ei saa kuidagi paaridesse jaotada.

\hnd
Selles ülesandes antakse punkte iga testi eest eraldi. Testid on jagatud gruppidesse, milles kehtivad järgmised lisatingimused:
\begin{xenum}
	\item (0 punkti) Ülesande tekstis olevad näited. Nende lahendamise eest punkte ei saa, aga nende hindamise tulemustest on näha, kas programm töötab serveris testides õigesti.
	\item (6 punkti) Iga soki värv on üks järgnevatest: \verb/punane/, \verb/kollane/, \verb/roheline/, \verb/sinine/.
	\item (10 punkti) $N \le 1\,000$.
	\item (4 punkti) Lisapiirangud puuduvad.
\end{xenum}

\end{yl}
\end{ol}
\end{document}
