\begin{yl}{2}{Sokid}{sokid}{1 sekund}{20 punkti}
  \emph{Idee: Tähvend Uustalu, teostus: Birgit Veldi, lahenduse selgitus: Birgit Veldi ja Ahto Truu}

  Juku tahab võimalikult vähe sokke üle värvida. Seega oleks tal mõistlik alustuseks ühte värvi sokkidest võimalikult palju paare moodustada. 
  Selleks võiks kokku lugeda, kui palju igat värvi sokke on. Siis on meil ühe värvi kohta kaks võimalust: 
  \begin{xitem}
    \item kui seda värvi sokke on paarisarv, saame need omavahel paarideks jaotada;
    \item kui seda värvi sokke on paaritu arv, saame kõik peale ühe omavahel paarideks jaotada.
  \end{xitem}
  Nii jääb meil igast sellisest värvist, mida oli paaritu arv, täpselt üks sokk üle.
  Kui neid ülejäänud sokke on paaritu arv, siis me neid kuidagi paarideks jaotada ei saa ning väljastada tuleb $-1$. (Seda võib tegelikult ka kohe algul kontrollida: kui sokkide koguarv $N$ on paaritu, siis neid paarideks jaotada pole võimalik.)
  Kui ülejäänud sokke on aga paarisarv, siis saame need ükskõik kuidas paarideks jaotada ning igas saadud paaris ühe soki oma paarilisega sobivaks üle värvida.

  Eelnev loogika näeb keeltes Python ja C++ programmeerituna välja umbes selline:

\begin{tabular}{p{\colwidth} p{\colwidth}}
Python:
\begin{lstlisting}[language=Python]
# loeme sokkide arvu
n = int(input())

# sokkide statistika värvide kaupa
sokid = {}
for i in range(n):
  varv = input().strip()
  if varv in sokid:
    # oleme seda värvi juba näinud,
    # suurendame loendurit
    sokid[varv] += 1
  else:
    # esimene seda värvi sokk,
    # alustame uue lenduriga 1-st
    sokid[varv] = 1

# leiame üksikute sokkide arvu 
yksi = 0
for arv in sokid.values():
  # paaritu arv lisab ühe üksiku soki
  yksi += arv % 2

# väljastame vastuse
if yksi % 2 > 0:
  # üksikuid on kokku paaritu arv,
  # lahendust ei ole
  print(-1)
else: 
  # peab pooled üksikud üle värvima
  print(yksi // 2)
\end{lstlisting}
&
C++:
\begin{lstlisting}[language=C++]
#include <iostream>
#include <string>
#include <map>
using namespace std;

int main() {
  // loeme sokkide arvu
  int n;
  cin >> n;

  // sokkide statistika värvide kaupa
  map<string, int> sokid;
  for (int i = 1; i <= n; ++i) {
    string varv;
    cin >> varv;
    // uue värviga sokid[varv]
    // poole pöördumine paneb selle
    // algväärtuseks automaatselt 0
    sokid[varv] += 1;
  }

  // leiame üksikute sokkide arvu 
  int yksi = 0;
  for (auto [varv, arv] : sokid) {
    // paaritu arv lisab ühe üksiku
    yksi += arv % 2;
  }

  // väljastame vastuse
  if (yksi % 2 > 0){
    // kui üksikuid on kokku paaritu
    // arv, siis lahendust ei ole
    cout << -1 << '\n';
  } else{
    // muidu peab pooled üle värvima
    cout << yksi / 2 << '\n';
  }
}
\end{lstlisting}
\end{tabular}

  Pythoni kasutajatel on võimalik lugeda värvide statistika kokku ka standardteegi Counter objekti abil:

\begin{lstlisting}[language=Python]
from collections import Counter

n = int(input())
loe = Counter(input().strip() for _ in range(n))

if n % 2 == 1:
  print(-1)
else:
  v = sum(c % 2 for c in loe.values())
  print(v // 2)
\end{lstlisting}

  Veel üks võimalus on pidada hulgatüüpi muutujas jooksvalt arvet ainult paariliseta sokkide üle:

\begin{tabular}{p{\colwidth} p{\colwidth}}
Python:
\begin{lstlisting}[language=Python]
n = int(input())

yksi = set()
for i in range(n):
  varv = input().strip()
  if varv in yksi:
    yksi.remove(varv)
  else:
    yksi.add(varv)

if n % 2 > 0:
  print(-1)
else: 
  print(len(yksi) // 2)
\end{lstlisting}
&
C++:
\begin{lstlisting}[language=C++]
#include <iostream>
#include <string>
#include <set>
using namespace std;

int main() {
  int n;
  cin >> n;

  set<string> yksi;
  for (int i = 1; i <= n; ++i) {
    string varv;
    cin >> varv;
    if (yksi.contains(varv)) {
      yksi.erase(varv);
    } else {
      yksi.insert(varv);
    }
  }

  if (n % 2 > 0){
    cout << -1 << '\n';
  } else{
    cout << yksi.size() / 2 << '\n';
  }
}
\end{lstlisting}
\end{tabular}

\end{yl}
