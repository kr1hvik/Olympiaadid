\begin{yl}{2}{Viie pildiga robotilõks}{captcha}{1 sekund}{30 punkti}
  \emph{Idee: Heno Ivanov, teostus ja lahenduse selgitus: Targo Tennisberg}

Lahendus koosneb kahest osast: väikevenna klõpsude läbisimuleerimine ja selle järel õige lahenduse leidmine. 

Programmeerimisaja kokkuhoiu mõttes on hea kasutada ühtset protseduuri klõpsude simuleerimiseks.
Vaja on hoida järjekorda märgitud piltidest, iga uus klõps kas eemaldab pildi järjekorrast või siis lisab selle järjekorra lõppu. 
Sellise protseduuri võib kirjutada ise või siis kasutada standardteegis olemasolet funktsionaalsust. 
Näiteks Pythoni loendil on juba olemas append ja remove meetodid, mis teevad just seda, mida vaja.

Kui väikevenna klõpsud on simuleeritud, tähistame lihtsuse mõttes pildid numbritega $1$ kuni $5$. 
Paneme tähele, et kui  pilt 1 ei ole kohe järjekorra alguses, tuleb kõik eelnevad pildid sealt niikuinii eemaldada.

Seega, kui järjekorra alguses on juba pilt $1$, on kõik korras. Vastasel korral eemaldame järjekorra algusest pildi ja lisame sellele vastava täiendava klõpsu vastusesse.
Kui pilt $1$ on leitud, hakkame uurima järjekorra teist positsiooni ning kordame sama protseduuri pildi $2$ jaoks. Seejärel teeme sama pildi $3$ jaoks jne.

Kui esialgne järjekord saab tühjaks, lisame kõik allesjäänud pildid õiges järjekorras vastusesse.
\end{yl}
