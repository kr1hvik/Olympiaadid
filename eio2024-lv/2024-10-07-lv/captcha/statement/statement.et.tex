% Idee: Heno Ivanov
% Tekst: Targo Tennisberg

\documentclass[a4paper,11pt]{article}
\usepackage[et]{../../eio}

\begin{document}

\begin{ol}{\eio}{\lv 07.--13.10.2024}{\yle}{}
  \begin{yl}{2}{Viie pildiga robotilõks}{captcha}{1 sekund}{30 punkti}
    Juku tahab minna veebisaidile, mis näitab talle viie pildiga robotilõksu ehk \textit{captcha}t.
    Saidile sissepääsemiseks peab Juku klõpsima kõiki pilte täpselt õiges järjekorras.
    Iga klõpsuga tekib pildi juurde number ($1, 2, \ldots$).
    Kui juba märgitud pilti uuesti klõpsata, kaob number ära ning kõik sellest suuremad numbrid muutuvad ühe võrra väiksemaks.

    Enne kui Juku jõuab midagi klõpsata, kutsub ema ta arvuti juurest ära. Naastes näeb Juku, et tema väikevend on juba erinevaid pilte klõpsinud.

    Leia, kuidas Juku saaks nüüd robotilõksu minimaalse klõpsude arvuga läbida.

    \sis Sisendi esimesel viiel real on igaühel $1$ kuni $10$ ladina suurtähtest koosnev sõne: piltide nimed selles järjekorras, milles neid tuleks klõpsida. Kõik viis sõnet on erinevad.

    Järgmisel real on täisarv $N$ ($0 \le N \le 10$): väikevenna tehtud klõpsude arv.

    Lõpuks on $N$ real igaühel üks sõne: nende piltide nimed, mida vend klõpsis (nende klõpsimise järjekorras).

    \val Väljundi esimesele reale kirjutada täisarv $M$: minimaalne klõpsude arv, mille Juku peab tegema.

    Järgmisele $M$ reale kirjutada nende piltide nimed, mida Juku peab klõpsima. Kui võimalusi on mitu, võib väljastada ükskõik millise neist.

    \nde[0]{3cm}{3cm}

    Saidile pääsemiseks on vaja saada piltide juurde järgmised numbrid: \verb/KOER/ --- $1$, \verb/PORGAND/ --- $2$, \verb/AUTO/ --- $3$, \verb/TROLL/ --- $4$, \verb/PIRN/ --- $5$.

    Väikevenna klõpsimise järel on seis selline: \verb/TROLL/ --- $1$, \verb/KOER/ --- $2$, \verb/PORGAND/ --- $3$, \verb/AUTO/ --- numbrita, \verb/PIRN/ --- numbrita.

    Kui Juku klõpsab pildil \verb/TROLL/, kaob selle number ära ja tulemus on selline: \verb/KOER/ --- $1$, \verb/PORGAND/ --- $2$, \verb/AUTO/ --- numbrita, \verb/PIRN/ --- numbrita, \verb/TROLL/ --- numbrita.
    Kui Juku klõpsab nüüd piltidel \verb/AUTO/, \verb/TROLL/ ja \verb/PIRN/, saavad need numbrid $3$, $4$ ja $5$ ning Juku pääsebki saidile.

    \clearpage
    \nde[1]{3cm}{3cm}

    Väikevenna klõpsimise järel on seis selline: \verb/ESIMENE/ --- $1$, \verb/TEINE/ --- $2$, \verb/KOLMAS/ --- numbrita, \verb/NELJAS/ --- numbrita, \verb/VIIES/ --- numbrita.

    \hnd Selles ülesandes antakse punkte iga testi eest eraldi. Testid on jagatud gruppidesse, milles kehtivad järgmised lisatingimused:
    \begin{xenum}
      \item (0 punkti) Ülesande tekstis olevad näited. Nende lahendamise eest punkte ei saa.
      \item (10 punkti) Väikevend tegi \textit{kokku} ülimalt ühe klõpsu.
      \item (10 punkti) Väikevend klõpsas \textit{iga pilti} ülimalt ühe korra.
      \item (10 punkti) Lisapiirangud puuduvad.
    \end{xenum}
  \end{yl}
\end{ol}

\end{document}
