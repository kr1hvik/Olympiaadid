% Idee: Olivia Tennisberg
% Tekst: Olivia Tennisberg

\documentclass[a4paper,11pt]{article}
\usepackage[et]{../../eio}

\begin{document}

\begin{ol}{\eio}{\lv 07.--13.10.2024}{\yle}{}
  \begin{yl}{4}{Kahepäevane võistlus}{voistlus}{1 sek / 3 sek}{60 punkti}
    Eriliselt Imeline Olümpiaad (EIO) on kahepäevane võistlus, kus osales $N$ õpilast.
    Võistluse esimese päeva tulemused on juba avalikult teada. Keegi lekitas ka
    teise päeva punktitabeli, aga seal pole õpilaste nimesid juures ning seega
    pole teada, kes õpilastest millise nendest skooridest sai.

    Eeldusel, et kõik õpilaste ja skooride vahelised vastavused on võimalikud, leida
    iga õpilase jaoks, mitmel juhul tema võistluse võidaks. Täpsemalt, leida, mitmel juhul
    oleks tema kahe päeva summaarne skoor kõigi õpilaste seas maksimaalne (loeb ka, kui
    ta jääb esimest kohta jagama).

    Kuna kombinatsioonide arv võib olla väga suur, väljastada tegeliku arvu
    asemel jääk, mis tekib selle jagamisel arvuga $10^9+7$.

    \sis Sisendi esimesel real on õpilaste arv $N$ ($1 \le N \le 300$).

    Sisendi teisel real on $N$ erinevat täisarvu $A_1, A_2, \ldots, A_N$ kahanevas
    järjekorras ($10^9 \ge A_1 > A_2 > \ldots > A_N \ge 0$), kus $A_i$ on $i$-nda õpilase
    esimese päeva skoor. 

    Sisendi kolmandal real on $N$ erinevat täisarvu $B_1, B_2, \ldots, B_N$ juhuslikus 
    järjekorras ($0 \le B_i \le 10^9$), kus $B_i$ on \textit{mingi} õpilase teise päeva skoor. 

    \val Väljastada $N$ täisarvu $C_1, C_2, \ldots, C_N$, kus $C_i$ on 
    $i$-nda õpilase võiduvõimaluste arv mooduli $10^9+7$ järgi, igaüks eraldi reale.

    \nde[0]{3cm}{3cm}

    Esimene õpilane võidab võistluse siis, kui ta saab teisel päeval 6 või 9 punkti ning teiste õpilaste skoorid seda rohkem ei mõjuta.

    Teine õpilane võidab võistluse mõlemas kombinatsioonis, kus ta saab teisel päeval 9 punkti.

    Kolmas õpilane võidab ainult siis, kui tema saab teisel päeval 9 punkti, teine õpilane saab 6 punkti ja esimene õpilane 2 punkti.

    \nde[1]{3cm}{3cm}

    Alati võidab õpilane, kes saab teisel päeval 10 punkti.

    \nde[2]{3cm}{3cm}

    \hnd Selles ülesandes on testid jagatud gruppidesse. Iga grupi eest saavad punkte
    ainult need lahendused, mis läbivad \textbf{kõik} sellesse gruppi kuuluvad
    testid. Gruppides kehtivad järgmised lisatingimused:

    \begin{minipage}[t]{0.45\textwidth}
      \begin{xenum}
        \item (0 punkti) Näited.
        \item (18 punkti) $N \le 10$.
      \end{xenum}
    \end{minipage}
    \begin{minipage}[t]{0.45\textwidth}
      \begin{xenum} \setcounter{enumi}{2}
        \item (18 punkti) $N \le 80$.
        \item (24 punkti) Lisapiirangud puuduvad.
      \end{xenum}
    \end{minipage}
  \end{yl}
\end{ol}

\end{document}
