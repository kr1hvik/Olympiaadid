\begin{yl}{4}{Kahepäevane võistlus}{voistlus}{1 sek / 3 sek}{60 punkti}
  \emph{Idee ja teostus: Olivia Tennisberg, lahenduse selgitus: Tähvend Uustalu}

  \subsection*{Alamülesanne $N \le 10$}

  Esimese alamülesande lahendamise jaoks proovime läbi kõik variandid panna
  igale esimese päeva skoorile vastavusse mõni teise päeva skoor. Selleks
  on $N! = 1 \cdot 2 \cdot 3 \cdots N$ varianti, mis $N = 10$ puhul jääb
  kolme ja poole miljoni ringi. Olgu öeldud, et nii Pythonis kui C++-s on variantide
  läbi vaatamiseks mugavaid abifunktsioone. Pythonis on üks kasulik funktsioon
  \verb/itertools.permutations/, C++-s on aga näiteks võimalik kasutada
  \verb/std::next_permutation/. Näiteks Pythonis võiks lahendus
  olla midagi sellist.
  \begin{lstlisting}[language=Python]
import itertools

N = int(input())
A = list(map(int, input().split()))
B = list(map(int, input().split()))

# answer[i] olgu i-nda õpilase võiduvõimaluste arv, esialgu 0
answer = [0] * N

# itertools.permutations(B) tagastab listi B kõik permutatsioonid
# seejuures kordustega, näiteks itertools.permutations([2, 1, 2])
# sisaldab (1, 2, 2) kaks korda
# see on ka see, mida me praegu tahame
for perm in itertools.permutations(B):
    # leiame, mis oleks sellise vastavuse korral võitja skoor
    # seda saab kindlasti ka mingi one-lineriga, aga jätame
    # praegu lahenduse lihtsaks
    highest_score = 0
    for i in range(N):
        highest_score = max(highest_score, A[i] + perm[i])

    # nüüd lisame ühe kõikide osalejate vastusele, kes selle
    # skoori saavutasid
    for i in range(N):
        if A[i] + perm[i] == highest_score:
            answer[i] += 1

print(*answer, sep="\n")
  \end{lstlisting}

  Selline lahendus teenib 18 punkti. Suuremates testides aga ületab
  ajalimiidi. Mis on selle põhjus? Ülal märkisime, et vaja on läbi vaadata
  $N!$ varianti. Kui $N = 80$, siis $N! \approx 7 \cdot 10^{118}$;
  kui $N = 300$, siis juba $N! \approx 3 \cdot 10^{614}$.

  Kui arvestada, et ühe variandi läbi vaatamiseks kulub 10 nanosekundit
  ehk $10^{-8}$ sekundit, siis $300!$ variandi läbi vaatamiseks kulub
  suurusjärgus $10^{599}$ aastat (mis on kõvasti rohkem, kui universumi
  eluiga mistahes hääbumise stsenaariumis).

  Raskemates ülesannetes on see tüüpiline. Peaaegu kõikidele informaatikaülesannetele
  on võimalik \emph{mingi} lahendus välja mõelda, näiteks kõikide variantide läbi
  vaatamise kaudu (kuigi mõnikord osutub ka sellise programmi kirjutamine üsna
  keeruliseks). Vajame lisaks toimivale lahendusele ka piisavalt kiiret lahendust.
  Mõnes ülesandes ongi see kiire lahenduse leidmine ülesande lahendamise lõviosa.
  Selle juurde käib oskus hinnata, kui kaua programm halvimal juhul aega võtab.
  Üldiselt ei maksa loota sellele, et testide seas seda kõige halvemat juhtu lihtsalt
  ei ole.

  Kindlasti ei piisa siin teatud pisioptimeeringutest, nagu C++-s \verb/i++/ asemel
  \verb/++i/ kirjutamine. Meie praegune lahendus ei ole mõne millisekundi võrra liiga
  aeglane, vaid õige mitu suurusjärku liiga aeglane. Täislahenduseks on vaja hoopis
  teistsugust lähenemist.

  \subsection*{Täislahendus}

  Sorteerime teise päeva tulemused kasvavas järjekorras.
  Edasi on täislahendust hea selgitada näite abil. Vaatleme järgnevat testi:
  \[
  \begin{array}{r|cccccc}
    A & 12 & 10 & 9 & 8 & 6 & 1 \\
    \hline
    B &  1 &  2 & 5 & 7 & 8 & 9
    \end{array}
  \]
  Uurime iga paari $(i, j)$ kohta, mitu võimalust on õpilasel $i$ võistlus
  võita, kui ta saab teisel päeval $j$-nda tulemuse.
  \begin{lstlisting}[language=Python]
N = int(input())
A = list(map(int, input().split()))
B = list(map(int, input().split()))
B.sort()
for i in range(N):
    for j in range(N):
        # TODO: mitu võimalust on i-ndal õpilasel võistlus võita
        # eeldusel, et ta saab teisel päeval j-nda tulemuse
  \end{lstlisting}
  Näiteks selgitame välja, mitu võimalust on võita õpilasel, kes sai esimesel
  päeval 8 punkti, eeldusel, et ta sai teisel päeval 7 punkti. See tähendab,
  et ükski ülejäänud õpilane ei tohi saada üle $8 + 7 = 15$ punkti.

  Ülejäänud õpilased on siis:
  \[
  \begin{array}{r|cccccc}
    A & 12 & 10 & 9 & 6 & 1 \\
    \hline
    B &  1 &  2 & 5 & 8 & 9
  \end{array}
  \]
  Proovime välja arvutada, mitu võimalust on igale ülejäänud õpilasele valida
  vastav teise päeva tulemus. Alustame sellest, kellel oli esimesel päeval
  kõige kõrgem tulemus, sest tal on kõige vähem variante.
  \begin{xitem}
  \item Õpilane, kes sai esimesel päeval 12 punkti, pidi teisel päeval saama
    1 või 2 punkti (kui ta oleks saanud juba 5 punkti, siis oleks tal kokku
    $12 + 5 = 17 > 15$ punkti). Seega 2 varianti.
  \item Õpilane, kes sai esimesel päeval 10 punkti, pidi teisel päeval saama
    1, 2 või 5 punkti. Üks nendest valikutest on aga juba eelmisele õpilasele
    reserveeritud. Seega $3 - 1 = 2$ varianti.
  \item Õpilane, kes sai esimesel päeval 9 punkti, pidi teisel päeval saama
    1, 2 või 5 punkti. Kaks nendest valikutest on eelmistele õpilastele reserveeritud.
    Kokku $3 - 2 = 1$ variant.
  \item Õpilane, kes sai esimesel päeval 6 punkti, pidi teisel päeval saama
    1, 2, 5, 8 või 9 punkti, neist kolm on eelmistele reserveeritud.
    Kokku $5 - 3 = 2$ varianti.
  \item Viimasel õpilasel on analoogiliselt $5 - 4 = 1$ variant.
  \end{xitem}
  Kokku on sellel õpilasel $2 \cdot 2 \cdot 1 \cdot 2 \cdot 1 = 8$ võimalust
  võita eeldusel, et ta sai teisel päeval 7 punkti.
  \begin{lstlisting}[language=Python]
MOD = 10**9 + 7
N = int(input())
A = list(map(int, input().split()))
B = list(map(int, input().split()))
B.sort()
for i in range(N):
    total_ways = 0
    for j in range(N):
        # mitu võimalust on i-ndal õpilasel võistlus võita
        # eeldusel, et ta saab teisel päeval j-nda tulemuse
        # selleks peavad ülejäänud õpilased saama ülimalt A[i] + B[j] punkti
        ways = 1

        # mitu sobivat teise päeva skoori on?
        good_scores = 0

        # hakkame kahanevas järjekorras õpilasi läbi käima, nagu ülal näidatud
        for k in range(N):
            if i == k:
                # õpilane, kes pidi võitma
                # tema jätame siinkohal vahele
                continue

            while good_scores < N and A[k] + B[good_scores] <= A[i] + B[j]:
                good_scores += 1

            # õpilasel k on good_scores võimalikku teise päeva skoori
            possible_scores = good_scores

            # aga mõned neist on juba eelmistele reserveeritud
            if k < i:
                possible_scores -= k
            else:
                possible_scores -= k - 1

            # lisaks võib good_scores sisse olla arvestatud ka j-s
            # mis on samuti juba ära lubatud
            if j < good_scores:
                possible_scores -= 1

            ways *= possible_scores
            ways %= MOD

        total_ways += ways
        total_ways %= MOD
    print(total_ways)
  \end{lstlisting}

  Keerukus on $O(N^3)$.

  \textbf{Lisaväljakutse:}
  Lahenda ülesanne keerukusega $O(N^2 \log N)$ (võib näiteks eeldada, et sisendi
  piirangutes on $N \le 1000$).

  \subsection*{Jäägiga jagamisest}

  Siia juurde on hea rääkida ka moodulitest. Selles ülesandes (ja üldse paljudes
  ``võimaluste loendamise'' ülesannetes) on palutud vastuse asemel väljastada
  jääk, mis tekib vastuse jagamisel arvuga $10^9 + 7$ (öeldakse ka: ``vastus
  mooduli $10^9 + 7$ järgi'').

  Esiteks olgu öeldud, et kõikides endast lugu pidavates programmerimiskeeltes
  on selline jäägi leidmine sisseehitatud. Nii Pythonis kui C++-s leiab
  \texttt{a \% b} jäägi, mis tekib arvu $a$ jagamisel arvuga $b$.\footnote{Selles
    ülesandes see küll suurt rolli mängida ei tohiks, aga
    negatiivsete arvudega tuleb kohati ettevaatlik olla: \texttt{-8 \% 5} on
    Pythonis \texttt{2}, C++-s aga \texttt{-3}. Mõlemad on omamoodi loogilised.}
  Käsitsi mingit kirjalikku jagamist ei ole kindlasti vaja programmeerida!

  Milleks aga sellist täiendavat sammu vaja on?
  Asi on selles, et vastus võib olla väga suur arv. Näiteks kui antud ülesandes
  $N = 300$ ja test on niisugune, et alati võidab üks ja sama õpilane, siis on
  vastus
  \[ N! = 1 \cdot 2 \cdot 3 \cdots N \approx 3 \cdot 10^{614}. \]
  C++-s on tavalised täisarvud 32- või 64-bitised, mis tähendab, et nende
  väärtus saab olla vastavalt ülimalt $2\,147\,483\,647$ või $9\,223\,372\,036\,854\,775\,807$.
  Need ülempiirid on palju väikesemad kui ülaltoodud 615-kohaline arv.
  Pythonis, kus tavaline täisarv saab olla kuitahes suur, on selles mõttes
  lihtsam, aga see lihtsus on omamoodi petlik: väga suurte arvudega arvutamine
  võib osutuda aeglaseks. Antud ülesandes ei ole see nii oluline, aga paljudes
  variantide loendamise ülesannetes ei ole vastused mitte mõnesaja-, vaid miljardikohalised.

  Paneme tähele,\footnote{Jagamisega on pisut keerulisem.
    Selles ülesandes ei ole see oluline, aga saab näidata, et kui $p$ on algarv ning
    $a$ jagub $b$-ga, siis $\frac{a}{b} \bmod{p} = (a b^{p - 2}) \bmod{p}$.} et
  \begin{align*}
    (a + b) \bmod{c} &= (a \bmod{c} + b \bmod{c}) \bmod{c} \\
    (a - b) \bmod{c} &= (a \bmod{c} - b \bmod{c}) \bmod{c} \\
    a \cdot b \bmod{c} &= [(a \bmod{c}) \cdot (b \bmod{c})] \bmod{c}.
  \end{align*}
  Siin tähistab $\bmod$ mooduli võtmise ehk jäägi leidmise tehet.
  See tähendab, et matemaatilises mõttes ei ole vahet, kas võtta moodul alles
  arvutuskäigu lõpus või pärast iga vahevastuse leidmist.

  Ülesannetes, kus vastus tuleb väljastada mooduli järgi, ongi tavaliselt
  mõeldud, et lahenduses võetakse moodulit iga tehte
  järel, sest nii ei kasva ka vahetulemused liiga suureks.
\end{yl}
