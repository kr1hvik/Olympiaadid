\begin{yl}{1}{Buss nr 6}{buss}{1 sekund}{20 punkti}
  \emph{Idee: Tähvend Uustalu, teostus ja lahenduse selgitus: Birgit Veldi}

  Selle ülesande lahendamiseks peame kõik peatused läbi vaatama ja iga peatuse juures kontrollima, kas Juku sinna peatusse joostes bussi peale jõuaks või mitte. Jõudmise korral peame lisaks leidma, kui palju tal aega varuks jääb ja mittejõudmise korral, kui palju tal aega puudu jääb.

  See kõik vajab aegade võrdlemist ja aegadega arvutamist. See oleks võimalik, kui aegu hoida tundide ja minutitena. Näiteks $J$ minuti liitmine ajale $H:M$ võiks välja näha umbes nii:
  \begin{lstlisting}[language=Python]
  M = M + J       # liidame J minutit
  H = H + M // 60 # kanname täistunnid üle tunninäitu
  M = M % 60      # ja võtame need minutinäidust maha
  \end{lstlisting}
  Kahe kellaaja lahutamine või võrdlemine oleks umbes sama tülikas.
  Hoopis lihtsam on kõik ajad teisendada minutiteks alates südaööst (1 tund = 60 min):
  \begin{lstlisting}[language=Python]
  T = 60 * H + M  # esitame aja H:M minutites
  \end{lstlisting}
  Siis saame edaspidi teha kõik arvutused ja võrdlused mugavalt täisarvudega. 

  Et teada saada, kas Juku jõuab bussi peale või mitte, peame võrdlema aega, mis kulub Jukul peatusse jõudmiseks, ajagapraegusest hetkest bussi sellest peatusest väljumiseni. Jukul kuluva aja saame sisendist. Bussil kuluva aja leidmiseks lahutame bussi väljumisajast praeguse kellaaja.

  Näitame, et ülesande lahendamiseks piisab leida maksimaalne vahe, mis tekib, kui lahutame bussil kuluvast ajast Jukul kuluva aja.

  Kui Jukul ei kulu rohkem aega kui bussil, siis ta jõuab bussi peale. Seega lahutades bussil kuluvast ajast Jukul kuluva aja, on tulemus mittenegatiivne ning see ongi aeg, mis Jukul üle jääb ning mida tahame maksimeerida.

  Kui Jukul kulub aga rohkem aega kui bussil, siis ta bussile ei jõua ning sama tehte vastus on negatiivne. Selle vastuse absoluutväärtus näitab, mitu minutit jääb Jukul bussi peale jõudmisest puudu.
 
  Et iga mittenegatiivne arv on suurem igast negatiivsest arvust, siis saab suurim bussil kuluva aja ja Jukul kuluva aja vahe olla negatiivne ainult siis, kui Juku ei jõua bussi peale üheski peatuses. Kuna negatiivne arv on seda suurem, mida väiksem on selle absoluutväärtus, siis neist negatiivsetest arvudest maksimum vastab peatusele, kus Jukul jääb bussi peale jõudmisest puudu kõige vähem.

  Selle idee elluviimiseks teeme kaks muutujat, millest üks peab meeles seni suurimat ajavahet ja teine seda peatust, kus selline vahe tekkis. Kuna bussil peatusesse minekuks kuluv aeg on mittenegatiivne ja Jukul saab kuluda maksimaalselt $10\,000$ minutit, ei saa see vahe kunagi olla väiksem kui $-10\,000$. Seega võtame algväärtuseks midagi sellest veel väiksemat, näiteks $-10\,001$.

  Seejärel käime kõik peatused läbi ja kui leiame suurema ajavaru, siis uuendame muutujaid.

  Sellised lahendused ongi toodud võistluse materjalide arhiivis alamkaustas \verb'buss/solution' failides \verb'sol.py' ja \verb'sol.cpp'.
\end{yl}
