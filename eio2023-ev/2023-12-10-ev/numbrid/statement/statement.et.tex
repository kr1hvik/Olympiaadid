% Idee: Heno Ivanov
% Tekst: Heno Ivanov

\documentclass[a4paper,11pt]{article}
\usepackage[et]{../../eio}

\begin{document}
\begin{ol}{\eio}{\ev 10.12.2023}{\yle}{}
\begin{yl}{2}{Numbrid}{numbrid}{1 sek / 3 sek}{20 punkti}

Värskelt asutatud jalgpalliklubi mängijad valivad oma särkidele numbreid. Iga mängija ütleb, millist numbrit ta soovib.

Kirjutada programm, mis leiab kõik sellised numbrid, mida mitte keegi ei soovinud, kuid mille korral leidub mängija, kes soovis väiksemat numbrit, ja mängija, kes soovis suuremat numbrit.

\sis 
Sisendi esimesel real on numbri valinud mängijate arv $N$ ($1 \le N \le 50\,000$).
Teisel real on $N$ tühikutega eraldatud täisarvu: mängijate soovitud numbrid. Kõik need numbrid on lõigust $0 \ldots 10\,000\,000$.

\val 
Väljundi esimesele reale kirjutada mittesoovitud numbrite koguarv $M$. Teisele reale väljastada need numbrid tühikutega eraldatult ja kasvavas järjestuses. Võib eeldada, et korrektse vastuse korral $M \le 50\,000$.

\nde[0]{4cm}{3cm}

\hnd Testides, mis annavad esimesed 10~punkti, on mängijate soovitud numbrid sisendis loetletud kasvavas järjekorras. Testides, mis annavad ülejäänud 10~punkti, lisapiiranguid ei ole.

\end{yl}
\end{ol}
\end{document}
