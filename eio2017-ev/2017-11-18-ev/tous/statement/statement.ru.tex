% Idee: Ahto Truu
% Tekst: Ahto Truu
% Tõlge: Aleksei Lissitsin

\documentclass[a4paper,11pt]{article}
\usepackage[ru]{../../eio}
\begin{document}

\begin{ol}{\eio}{\ev 18.11.2017}{\yle}{}
\begin{yl}{1}{Максимальный подъём}{tous}{1 секунда}{20 очков}

Одна из важнейших характеристик лыжных, беговых и велосипедных марафонов --- это их высотный профиль и особенно его максимальный подъём.

Высотный профиль трассы даёт высоты $H_1$, $H_2$, \dots, $H_N$ для $N$ точек трассы. Подъёмом называется такая последовательность подряд идущих точек, в которой каждая следующая точка строго выше предыдущей. Высотой подъёма называется разница высот конца и начала подъёма.

Написать программу, которая найдёт высоту максимального подъёма по заданному высотному профилю.

\sis Первая строка текстового файла \sisf содержит число точек высотного профиля $N$ ($1 \le N \le 50\,000$), а следующие $N$ строк содержат целочисленные высоты этих точек $H_i$ ($0 \le H_i \le 1\,000\,000$) в порядке от старта к финишу.

\val На единственную строку текстового файла \valf вывести одно целое число: высоту максимального подъёма, т.е. максимальную разность высот $H_i - H_j$, где $j \le i$ и $H_j < H_{j+1} < \dots < H_{i-1} < H_i$. Если на трассе нет ни одного подъёма (хаха :), то вывести 0.

\nde[0]{3cm}{3cm}

В приведённом примере два подъёма, один высотой 2 и второй высотой 3:

\joon{0}

\hnd В тестах общей стоимостью 10 очков выполняется дополнительное условие $N \le 50$.

\end{yl}
\end{ol}
\end{document}
