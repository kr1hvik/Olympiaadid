% Idee: Oliver-Matis Lill
% Tekst: Oliver-Matis Lill
% Tõlge: Aleksei Lissitsin

\documentclass[a4paper,11pt]{article}
\usepackage[ru]{../../eio}
\begin{document}

\begin{ol}{\eio}{\ev 18.11.2017}{\yle}{}
\begin{yl}{3}{Настольная игра}{mang}{1 секунда}{40 очков}

Недавно ты получил в подарок настольную игру, похожую на Лилу (``Змеи и лестницы''), но цикличную.

На игровом поле есть $N$ упорядоченных клеток $1, \ldots, N$, причём за клеткой $N$ снова идёт клетка $1$. В каждой клетке $i$ записано какое-то целое число $a_i$. Если $a_i = 0$, то, будучи на клетке $i$, игрок должен кинуть шестигранный кубик и передвинуться на выпавшее число клеток вперёд. Если $a_i \ne 0$, то игрок должен передвинуться на $a_i$ клеток вперёд (назад, если $a_i$ отрицательно); это продолжается, пока игрок не прибудет на клетку, в которой записан $0$ (но можно угодить и в бесконечный цикл). Игра начинается с клетки $1$, причём известно, что $a_1 = 0$.

После изучения игры у тебя появилось подозрение, что есть такие клетки, куда невозможно попасть. Написать программу, которая найдёт все те клетки, которые можно посетить, играя в игру.

\sis Первая строка текстового файла \sisf содержит число клеток игрового поля $N$ ($1 \le N \le 1000$).
Вторая строка содержит $N$ разделённых пробелами целых чисел $a_1, \ldots, a_N$ ($-N < a_i < N$, $a_1 = 0$).

\val На единственную строку текстового файла \valf вывести $N$ разделённых пробелами чисел $0$ или $1$. Стоящее на позиции $i$ число $1$ означает, что в клетку $i$ попасть можно, а число $0$, что туда попасть нельзя.

\nde[0]{5cm}{5cm}

\nde[1]{5cm}{5cm}

\end{yl}
\end{ol}
\end{document}
