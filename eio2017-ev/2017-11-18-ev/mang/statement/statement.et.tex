% Idee: Oliver-Matis Lill
% Tekst: Oliver-Matis Lill

\documentclass[a4paper,11pt]{article}
\usepackage[et]{../../eio}
\begin{document}

\begin{ol}{\eio}{\ev 18.11.2017}{\yle}{}
\begin{yl}{3}{Lauamäng}{mang}{1 sekund}{40 punkti}

Sa said hiljuti kingituseks lauamängu, mis on nagu Tsirkus, aga tsükliline.

Mängulaual on $N$ järjestatud ruutu $1, \ldots, N$, kusjuures ruudule $N$ järgneb ruut $1$. Igale ruudule $i$ on märgitud mingi täisarv $a_i$. Kui $a_i = 0$, siis ruudul $i$ olles peab mängija viskama kuuetahulist täringut ja liikuma saadud tulemuse võrra edasi. Kui $a_i \ne 0$, peab mängija liikuma $a_i$ võrra edasi (tagasi, kui $a_i$ on negatiivne); see kordub, kuni mängija jõuab ruudule, millel on kirjas $0$ (aga on võimalik sattuda ka lõpmatusse tsüklisse). Mäng algab ruudult $1$ ja on teada, et $a_1 = 0$.

Mängu vaadates tekkis Sul kahtlus, et on ruute, kuhu ei olegi võimalik kunagi sattuda. Kirjuta programm, mis leiab, milliseid ruute on võimalik mängu jooksul külastada.

\sis Tekstifaili \sisf esimesel real on mängulaua ruutude arv $N$ ($1 \le N \le 1000$).
Teisel real on $N$ tühikutega eraldatud täisarvu $a_1, \ldots, a_N$ ($-N < a_i < N$, $a_1 = 0$).

\val Tekstifaili \valf ainsale reale väljastada $N$ tühikutega eraldatud arvu $0$ või $1$. Positsioonil $i$ olev arv $1$ tähendab, et laua ruudule $i$ on võimalik sattuda, ja arv $0$, et sinna ei ole võimalik sattuda.

\nde[0]{5cm}{5cm}

\nde[1]{5cm}{5cm}

\end{yl}
\end{ol}
\end{document}
