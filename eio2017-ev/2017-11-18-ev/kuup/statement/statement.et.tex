% Idee: Heno Ivanov
% Tekst: Heno Ivanov

\documentclass[a4paper,11pt]{article}
\usepackage[et]{../../eio}
\usepackage{wrapfig}
\begin{document}

\begin{ol}{\eio}{\ev 18.11.2017}{\yle}{}
\begin{yl}{4}{Kuubi kleepimine}{kuup}{1 sek}{50 punkti}

Jukul on vaja valmistada paberist kuup. Paberile on juba joonestatud $10 \times 10$ ruudustik, mille alumise vasaku ruudu koordinaadid on $(1,1)$ ja ülemise parema omad $(10,10)$. Juku valib 6 paarikaupa erinevat ruutu.

Kontollida, kas valitud ruutudest on võimalik kokku voltida kuup (lõigata ja voltida tohib ainult mööda olemasolevaid jooni).

Lisaks leida, milliste ruutude servadesse tuleks jätta kleepimiseks ribad. Kahest omavahel kokkupuutuvast lõikeservast tuleb riba jätta täpsele ühele. Ribad võivad asuda ka ruudustiku servas.

\sis Tekstifail \sisf sisaldab täpselt 6 rida, igal real kaks tühikuga eraldatud täisarvu: Juku valitud ruutude koordinaadid $x_i$ ja $y_i$ ($1 \le x_i \le 10$, $1 \le y_i \le 10$).

\val Tekstifaili \valf esimesele reale väljastada \verb'JAH', kui Juku valitud ruutude välja\-lõikamisel moodustub (sidus) kujund, millest on võimalik kokku voltida kuup, vastasel juhul \verb'EI'. Kui kuubi voltimine on võimalik, väljastada järmistele ridadele programmi poolt leitud kleepimis\-ribade asukohad: tühikutega erladatult ruudu X- ja Y-koordinaadid ning suund (\verb'N' --- üles, \verb'E' --- paremale, \verb'S' --- alla, \verb'W' --- vasakule).

\begin{wrapfigure}{r}{0.3\textwidth}
\vspace{-10pt}
\joon{0}
\vspace{+10pt}
\joon{1}
\vspace{-10pt}
\end{wrapfigure}

\nde[0]{3cm}{3cm}

\nde[1]{3cm}{3cm}

\hnd Selles ülesandes on testid jagatud gruppidesse. Igas grupis saab 50\% punktidest programm, mis annab kõigis selle grupi testides korrektse \verb'JAH'/\verb'EI' vastuse. Ülejäänud 50\% saab kleepimis\-ribade asukohtade eest.

\end{yl}
\end{ol}
\end{document}
