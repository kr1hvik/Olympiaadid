% Idee: Heno Ivanov
% Tekst: Heno Ivanov
% Tõlge: Aleksei Lissitsin

\documentclass[a4paper,11pt]{article}
\usepackage[ru]{../../eio}
\usepackage{wrapfig}
\begin{document}

\begin{ol}{\eio}{\ev 18.11.2017}{\yle}{}
\begin{yl}{4}{Склейка куба}{kuup}{1 секунда}{50 очков}

Вове нужно сделать куб из бумаги. На бумаге уже нарисована сетка $10 \times 10$, в которой у нижней левой клетки координаты $(1,1)$, а у верхней правой $(10,10)$. Вова выбирает 6 попарно различных клеток.

Проверить, можно ли из выбранных клеток сложить куб (резать и сгибать можно только вдоль уже имеющихся линий).

Дополнительно найти те края клеток, на которых нужно расположить полосы для склеивания. Из двух соприкасающихся краёв полосу нужно расположить ровно на одном. Полосы могут быть и на краях сетки.

\sis Текстовый файл \sisf содержит ровно 6 строк, на каждой строке по два разделённых пробелом целых числа: координаты одной из выбранных Вовой клеток $x_i$ и $y_i$ ($1 \le x_i \le 10$, $1 \le y_i \le 10$).

\val На первую строку текстового файла \valf вывести \verb'JAH', если при вырезании выбранных Вовой клеток образуется (связная) фигура, из которой можно сложить куб, в противном случае \verb'EI'. Если сложить куб можно, вывести на следующие строки найденные программой расположения клейких полос: разделённые пробелами X- и Y-координаты клетки и направление (\verb'N' --- вверх, \verb'E' --- направо, \verb'S' --- вниз, \verb'W' --- налево).

\begin{wrapfigure}{r}{0.3\textwidth}
\vspace{-10pt}
\joon{0}
\vspace{+10pt}
\joon{1}
\vspace{-10pt}
\end{wrapfigure}

\nde[0]{3cm}{3cm}

\nde[1]{3cm}{3cm}

\hnd В этом задании тесты разбиты на группы. В каждой группе 50\% очков получит программа, которая для всех тестов этой группы выведет правильный ответ \verb'JAH'/\verb'EI'. Остальные 50\% даются за правильные расположения клейких полос.

\end{yl}
\end{ol}
\end{document}
