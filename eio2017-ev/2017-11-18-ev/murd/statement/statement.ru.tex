% Idee: Rein Prank
% Tekst: Rein Prank, Sandra Schumann
% Tõlge: Aleksei Lissitsin

\documentclass[a4paper,11pt]{article}
\usepackage[ru]{../../eio}
\begin{document}

\begin{ol}{\eio}{\ev 18.11.2017}{\yle}{}
\begin{yl}{2}{Вычитание дробей}{murd}{1 секунда}{30 очков}

Написать программу, которая умеет вычитать обыкновенные дроби.

\sis Первая и вторая строки текстового файла \sisf содержат дроби $a/b$ и $c/d$, где $a$ и $c$ неотрицательные, а $b$ и $d$ положительные целые числа величиной до $1000$. Найти разность $a/b-c/d$ в виде несократимой простой или смешанной дроби. Вывести результат как в неформатированном, так и в форматированном виде.

\val На первую строку текстового файла \valf вывести целую часть результата (которая может оказаться и 0). Если дробная часть отличается от нуля, то вывести её на вторую строку в виде несократимой дроби $x/y$. Если дробная часть равна нулю, то оставить вторую строку файла пустой.

Если дробная часть отличается от нуля, то вывести результат и в форматированном виде. На третью строку файла вывести числитель дробной части, на четвёртую --- целую часть и дробную черту, составленную из знаков минуса, на пятую --- знаменатель дробной части. Длина дробной черты должна быть равна длине записи знаменателя. Записи числителя и знаменателя должны заканчиваться на той же позиции в строке, что и дробная черта. В форматированном виде целую часть, равную нулю, не выводить.

Если результат отрицательный, то обозначить это с помощью минуса перед целой частью. При отрицательном результате и отсутствии целой части вывести в неформатированном виде целую часть как \verb'-0', а в форматированном виде поставить минус перед дробью без числа \verb'0' (см. пример). При неотрицательном результате знак не выводить.

Все результаты вывести без пробелов, за исключением пробелов в началах третьей и пятой строк, необходимых для выравнивания записи числителя и знаменателя.

\nde[0]{3cm}{3cm}

\nde[1]{3cm}{3cm}

\nde[2]{3cm}{3cm}

\hnd Программа, которая верно выводит только неформатированный результат, получает 50\% от стоимости теста.

\end{yl}
\end{ol}
\end{document}
