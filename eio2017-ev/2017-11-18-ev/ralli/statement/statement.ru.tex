% Idee: Tähvend Uustalu
% Tekst: Taavet Kalda, Tähvend Uustalu
% Tõlge: Aleksei Lissitsin

\documentclass[a4paper,10pt]{article}
\usepackage[ru]{../../eio}
\begin{document}

\begin{ol}{\eio}{\ev 18.11.2017}{\yle}{}
\begin{yl}{5}{Ралли в глубоком космосе}{ralli}{1 сек / 10 сек}{60 очков}

По формуле Циолковского общая масса топлива, необходимого для ускорения ракеты из положения покоя до скорости $v$, равна
\[
  m = m_0\left(e^{\frac{v}{u}} - 1\right),
\]
где $m_0$ --- это масса пустой ракеты и $u$ --- скорость выброса топлива. Формула предполагает, что в период ускорения топливный бак опустошается до конца.

В рамках данного задания будем считать, что топливный бак имеет бесконечную вместимость, $m_0 = 1$, $u = 1$, $e \approx 2$ и $e^{\frac{v}{u}} \gg 1$. В этом случае на ускорение ракеты до скорости $v$ уходит $2^{v}$ единицы топлива.

В космосе проводится ралли, в котором есть $V$ контрольных пункта и $E$ трассы с препятствиями, соединяющих контрольные пункты. Известно, что для прохождения $k$-той трассы нужно ускорить ракету до скорости $k$.

Для прохождения контрольного пункта ракета должна полностью остановиться, причём на торможение топливо не тратится. В контрольных пунктах можно наполнить топливный бак ракеты.

Также известно, что ни одну пару контрольных пунктов не объединяет более чем одна трасса, ни одна трасса не объединяет какой-либо контрольный пункт с самим собой, и из каждого контрольного пункта существует путь по трассам в любой другой.

Ралли состоит из $Q$ этапов, на каждом этапе нужно двигаться из контрольного пункта $p$ в контрольный пункт $q$. Найти расход топлива, необходимый для каждого этапа. Так как расходы топлива могут оказаться очень большие, вывести их по модулю $10^9 + 7$.

\sis Первая строка текстового файла \sisf содержит три разделённых пробелами целых числа: число контрольных пунктов $V$~($1 \le V \le 10^5$), число трасс $E$~($1 \le E \le 3 \cdot 10^5$) и число этапов $Q$~($1 \le Q \le 10^5$).

Каждая из следующих $E$ строк содержит два разделённых пробелом целых числа $a$~и~$b$~($1 \le a \le V$, $1 \le b \le V$), которые показывают, что контрольные пункты $a$ и $b$ соединяются трассой, которую можно пройти в обоих направлениях. На строке $k + 1$ описывается трасса под номером $k$.

Каждая из следующих $Q$ строк содержит два разделённых пробелом целых числа $p$~и~$q$~($1 \le p \le V$, $1 \le q \le V$), которые показывают, в каких контрольных пунктах этап начинается и заканчивается.

\val В текстовый файл \valf вывести $Q$ строк, на каждую строку --- минимальный расход топлива на один этап. Расходы топлива за этапы вывести в том же порядке, в котором были даны этапы во входных данных.

\nde[0]{3cm}{3cm}

\hnd В этом задании тесты разбиты на группы. За каждую группу получат очки только те программы, которые пройдут все тесты этой группы. В группах выполняются следующие дополнительные ограничения:

\begin{enumerate}
\item $V \le 30$, $E \le 30$ (15 очков).
\item $Q = 1$, $V \le 10^3$, $E \le 10^3$ (15 очков).
\item Дополнительных ограничений нет (30 очков).
\end{enumerate}

\end{yl}
\end{ol}
\end{document}
