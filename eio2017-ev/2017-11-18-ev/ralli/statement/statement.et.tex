% Idee: Tähvend Uustalu
% Tekst: Taavet Kalda, Tähvend Uustalu

\documentclass[a4paper,10pt]{article}
\usepackage[et]{../../eio}
\begin{document}

\begin{ol}{\eio}{\ev 18.11.2017}{\yle}{}
\begin{yl}{5}{Ralli süvakosmoses}{ralli}{1 sek / 10 sek}{60 punkti}

Vastavalt Tsiolkovski valemile kulub raketi kiirendamiseks paigalseisust kiiruseni $v$ kütust kogumassiga
\[
  m = m_0\left(e^{\frac{v}{u}} - 1\right),
\]
kus $m_0$ on raketi tühimass ja $u$ kütuse heitekiirus. Valem töötab tingimusel, et kiirendamise käigus tehakse kütusepaak tühjaks.

Selles ülesandes eeldame, et raketi kütusepaak on lõputu mahuga, $m_0 = 1$, $u = 1$, $e \approx 2$ ja $e^{\frac{v}{u}} \gg 1$. Sel juhul kulub raketi kiirendamiseks kiiruseni $v$ kütust $2^{v}$ ühikut.

Kosmoses korraldatakse ralli, mis koosneb $V$ kontrollpunktist ja $E$ kahesuunalisest takistus\-rajast,
mis ühendavad kontrollpunkte. Takistusraja number $k$ läbimiseks on vaja kiirendada rakett kiiruseni $k$.

Iga kontrollpunkti läbimiseks peab rakett täielikult peatuma, kusjuures pidurdamine kütust ei kuluta.
Kontrollpunktides on võimalik raketi kütusepaaki täita.

Lisaks on teada, et ühtki kontroll\-punktide paari ei ühenda rohkem kui üks takistusrada,
ükski takistusrada ei ühenda mõnda kontrollpunkti iseendaga ja igast kontrollpunktist
pääseb mööda takistusradu igasse teise kontrollpunkti.

Ralli koosneb $Q$ etapist, igas etapis on vaja liikuda mingist kontrollpunktist $p$ mingisse kontroll\-punkti $q$. Leida iga etapi läbimiseks vajalik kütusekulu. Kuna kütusekulud võivad olla väga suured, väljastada
nad mooduli $10^9 + 7$ järgi.

\sis Tekstifaili \sisf esimesel real on kolm tühikutega eraldatud täisarvu: kontroll\-punktide arv $V$~($1 \le V \le 10^5$), takistusradade arv $E$~($1 \le E \le 3 \cdot 10^5$) ning etappide arv $Q$~($1 \le Q \le 10^5$).

Järgmisel $E$ real on igaühel kaks tühikuga eraldatud täisarvu $a$~ja~$b$~($1 \le a \le V$, $1 \le b \le V$), mis näitavad, et kontrollpunktid $a$ ja $b$ on ühendatud kahesuunalise takistusrajaga. Faili real number $k + 1$ kirjeldatakse takistusrada number $k$.

Järgmisel $Q$ real on igaühel kaks tühikuga eraldatud täisarvu $p$~ja~$q$~($1 \le p \le V$, $1 \le q \le V$), mis näitavad, mis kontroll\-punktides etapp vastavalt algab ja lõppeb.

\val Tekstifaili \valf väljastada $Q$ rida, igale reale ühe etapi läbimise minimaalne kütusekulu. Etappide kütusekulud väljastada samas järjekorras, milles etapid sisendis anti.

\nde[0]{3cm}{3cm}

\hnd Selles ülesandes on testid jagatud gruppidesse. Iga grupi eest saavad punkte ainult need lahendused, mis läbivad kõik sellesse gruppi kuuluvad testid. Gruppides kehtivad järgmised lisatingimused:

\begin{enumerate}
\item $V \le 30$, $E \le 30$ (15 punkti).
\item $Q = 1$, $V \le 10^3$, $E \le 10^3$ (15 punkti).
\item Lisapiirangud puuduvad (30 punkti).
\end{enumerate}

\end{yl}
\end{ol}
\end{document}
